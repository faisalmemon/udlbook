% Problem 7.1

\documentclass{article}

\usepackage{geometry}
\usepackage{amsmath}

% Blackboard letters
\usepackage{amsfonts}
\usepackage{graphicx}
\usepackage{breqn} 
\usepackage{verbatim}
% Hyper links
\usepackage{hyperref}

\title{Problem 7.1}

\begin{document}

\maketitle

\section{Problem}

A two-layer network with two hidden units in each layer can be defined as:

\begin{align*}
    y = \phi_{0}
    &+ \phi_{1}a
     \left[
        \nu_{01}
        + \nu_{11}a[\theta_{01} + \theta_{11}x]
        + \nu_{21}a[\theta_{02} + \theta_{12}x]
      \right]
      \\
    &+ \phi_{2}a
    \left[
        \nu_{02}
        + \nu_{12}a[\theta_{01} + \theta_{11}x]
        + \nu_{22}a[\theta_{02} + \theta_{12}x]
    \right]
\end{align*}

Compute the derivatives of the output $y$ with respect to each of the $13$ parameters directly (not using back propagation).  Use $\mathbb{I}$ for $ \frac{\partial{a[z]}}{\partial{z}}$ where $a$ represents the ReLU activation function.

\section{Answer}

\begin{align*}
    \frac{\partial{y}}{\partial{\phi_{0}}} &=
    1
    \\
    \frac{\partial{y}}{\partial{\phi_{1}}} &=
    a
    (
        \nu_{01} +\nu_{01}
        + \nu_{11}a[\theta_{01} + \theta_{11}x]
        + \nu_{21}a[\theta_{02} + \theta_{12}x]
    ) 
    \\
    \frac{\partial{y}}{\partial{\phi_{2}}} &=
    a
    (
        \nu_{02}
        + \nu_{12}a[\theta_{01} + \theta_{11}x]
        + \nu_{22}a[\theta_{02} + \theta_{12}x]
    ) 
\end{align*}

For the $\theta_{01}$ partial derivative I used chain rule twice, with $u_{1}$ as $\phi_{1}\nu_{11}a[\theta_{01}]$ for the first symmetric part of the equation $\phi_{1}$ (and similarly $u_{2}$ as  $\phi_{2}\nu_{12}a[\theta_{01}]$ for the $\phi_{2}$)
\begin{align*}
    y &= a[u_{1}] + a[u_{2}]
    \\
    \frac{\partial{y}}{\partial{\theta_{01}}} &=
    \frac{\partial{y}}{\partial{u_{1}}} \cdot \frac{\partial{u_{1}}}{\partial{\theta_{01}}} +  
    \frac{\partial{y}}{\partial{u_{2}}} \cdot \frac{\partial{u_{2}}}{\partial{\theta_{01}}}
    \\
    \frac{\partial{y}}{\partial{\theta_{01}}} &=
    \mathbb{I}(\phi_{1}\nu_{11}a[\theta_{01}]) \cdot \phi_{1}\nu_{11} + \mathbb{I}(\phi_{2}\nu_{12}a[\theta_{01}]) \cdot \phi_{2}\nu_{12}
\end{align*}

I dropped the final $a$ terms around $\theta_{01}$ to leave $ \cdot \phi_{1}\nu_{11}$ and $ \cdot \phi_{2}\nu_{12}$ because of the multiplication with the $\mathbb{I}$ covers the negative case as zero already.

I haven't yet worked out the other terms but I think similar logic applies.

I have asked for a sanity check on \href{https://math.stackexchange.com/questions/4914949/}{\texttt{math.stackexchange.com/questions/4914949}}.


\end{document}